\documentclass[12pt]{report}

\usepackage{amsmath}
\usepackage{amssymb}
\usepackage{geometry}
\usepackage{graphicx}
\usepackage{float}
\usepackage{bm}


\setlength\parindent{0pt}
\geometry{
	paper=letterpaper, % Change to letterpaper for US letter
	inner=1in, % Inner margin
	outer=1in, % Outer margin
	top=1.5in, % Top margin
	bottom=1in, % Bottom margin
	%showframe, % Uncomment to show how the type block is set on the page
}

\begin{document}
\begin{center}
\LARGE \bf RD+Mechanics Formulation
\end{center}

\section*{Reaction Diffusion}
\[ \frac{\partial\phi_T}{\partial t} = \nabla\cdot D_T \nabla \phi_T + \alpha_T\phi_T(1-\phi_T)\]

Where: 
\[ \phi_T = \textrm{Tumor volume fraction or number of tumor cells} \]
And parameters $D_T$ and $\alpha_T$ can be functions of a deformation measure:
\begin{align*}
D_T &= \overline{D}_T\exp(-\gamma_T^{Pa}J_T) \\
\alpha_T &= \overline{\alpha}_T\exp(-\gamma_T^{mob}J_T)
\end{align*}

\section*{Elastic Deformation}
Let $\bm{x} = \chi(\bm{X},t)$ be the current (deformed) configuration, a function of the reference (undeformed) configuration $\bm{X}$ and time. The deformation gradient is:
\[ \bm{F} = \frac{\partial \chi}{\partial \bm{X}} \]
And it can be decomposed into:
\[ \bm{F} = \bm{F}^S\bm{F}^G \]
Where $\bm{F}^G$ is the right stretch tensor, 
\[ \bm{F}^G = \lambda^G\bm{I} = (\beta\phi_T+1)\bm{I} \]

\subsection*{Small deformations}
When $||\nabla\bm{u}|| << 1$, the formulation can be rewritten in terms of infinitesimal strain,
\[ \bm{E}_T = \frac{1}{2}(\nabla\bm{u}_T+\nabla\bm{u}_T^T) \]
This can be decomposed:
\[ \bm{E}_T = \bm{E}_T^S+\bm{E}_T^G \]
$\bm{F}^G=\lambda^G\bm{I}$ is equivalent to the following growth strain:
\begin{align*}
\bm{E}_T^G &= \frac{1}{2}((\bm{F}_T^G)^T+\bm{F}_T^G)-\bm{I} \\
&= \frac{1}{2}(\lambda^G\bm{I}+\lambda^G\bm{I})-\bm{I} \\
&= (\lambda^G-1)\bm{I}  \\
\bm{E}_T^G &= \beta\phi_T\bm{I} 
\end{align*}
The Cauchy stress tensor is:
\[ \bm{T}_T = \bm{C}_T(\bm{E}_T-\bm{E}_T^G) = \bm{C}_T\bm{E}_T^S \]
Where $\bm{C}_T$ is the 4th order elasticity tensor. Under linear and isotropic assumptions:
\begin{align*}
\bm{T}_T &= 2G\bm{E}^S+\frac{2G}{1-2\nu}(\textrm{tr}\bm{E}^S)\bm{I} \\
&=2G(\bm{E}-\beta\phi_T\bm{I})+\frac{2G}{1-2\nu}\bm{I}(\textrm{tr}\bm{E}-\textrm{tr}\bm{E}^G) 
\end{align*}
Where $\textrm{tr}\bm{E}^G = 3\beta\phi_T$. Thus, for linear elasticity:
\begin{align*}
\nabla\cdot\bm{T}_T &= 0 \\
\bm{T}_T &= 2G(\bm{E}-\beta_T\phi_T\bm{I})+\frac{2G}{1-2\nu}(\textrm{tr}\bm{E}-3\beta_T\phi_T)\bm{I}
\end{align*}

\subsection*{Large Deformations: Hyperelasticity}
Compressible Neo-Hookean strain energy:
\begin{align*}
W &= \frac{G}{2}(I_{C_1}^S-3)+\frac{K}{2}(J^S-1)^2 \\
I_{C_1}^S &= \textrm{tr}(\bm{C}^S) \\
J^S &= \sqrt{\det(\bm{C}^S)} = \det(\bm{F}^S)
\end{align*}
1st PK stress: 
\begin{align*}
\bm{T} &= \frac{\partial W}{\partial \bm{F}} = \frac{\partial W}{\partial \bm{F}^S} \frac{\partial \bm{F}^S}{\partial \bm{F}} \\
\frac{\partial W}{\partial \bm{F}^S} &= \frac{G}{(J^S)^{5/3}}(\bm{B}^S-\frac{1}{3}\textrm{tr}(\bm{B}^S)\bm{I})+K(J^S-1)\bm{I} \\
\frac{\partial \bm{F}^S}{\partial \bm{F}} &= \bm{I}\otimes(\bm{F}^G)^{-1} = \frac{1}{\lambda^G}
\end{align*}
For nonlinear (Neo-Hookean Hyperelasticity):
\begin{align*}
\nabla \cdot \bm{T}_T &= 0 \\
\bm{T}_T &= \frac{1}{1+\beta\phi_T}\left[ \frac{G}{(J^S)^{5/3}}(\bm{B}^S-\frac{1}{3}\textrm{tr}(\bm{B}^S\bm{I})+K(J^S-1)\bm{I} \right]
\end{align*}












\end{document}